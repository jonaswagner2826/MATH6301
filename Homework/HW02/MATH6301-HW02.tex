% Standard Article Definition
\documentclass[]{article}

% Page Formatting
\usepackage[margin=1in]{geometry}
\setlength\parindent{0pt}

% Graphics
\usepackage{graphicx}

% Math Packages
\usepackage{physics}
\usepackage{amsmath, amsfonts, amssymb, amsthm}
\usepackage{mathtools}

% Extra Packages
\usepackage{pdfpages}
\usepackage{hyperref}
% \usepackage{listings}

% Section Heading Settings
\usepackage{enumitem}
% \renewcommand{\theenumi}{\alph{enumi}}
\renewcommand*{\thesection}{Problem \arabic{section}}
\renewcommand*{\thesubsection}{\alph{subsection})}
\renewcommand*{\thesubsubsection}{}%\quad \quad \roman{subsubsection})}

\newcommand{\Problem}{\subsubsection*{\textbf{PROBLEM:}}}
\newcommand{\Solution}{\subsubsection*{\textbf{SOLUTION:}}}
\newcommand{\Preliminaries}{\subsubsection*{\textbf{PRELIMINARIES:}}}

%Custom Commands
\newcommand{\N}{\mathbb{N}}
\newcommand{\Z}{\mathbb{Z}}
\newcommand{\Q}{\mathbb{Q}}
\newcommand{\R}{\mathbb{R}}
\newcommand{\C}{\mathbb{C}}

\newcommand{\Rel}{\mathcal{R}}

% \newcommand{\toI}{\xrightarrow{\textsf{\tiny I}}}
% \newcommand{\toS}{\xrightarrow{\textsf{\tiny S}}}
% \newcommand{\toB}{\xrightarrow{\textsf{\tiny B}}}

\newcommand{\divisible}{ \ \vdots \ }
\newcommand{\st}{\ : \ }

% Theorem Definition
\newtheorem{definition}{Definition}
\newtheorem{assumption}{Assumption}
\newtheorem{theorem}{Theorem}
\newtheorem{lemma}{Lemma}
\newtheorem{proposition}{Proposition}
% \newtheorem{example}{Example}
% \newtheorem{counterExample}{Counter Example}


%opening
\title{MATH 6301 Real Analysis I \\ Homework 2}
\author{Jonas Wagner\\ jonas.wagner@utdallas.edu}
\date{2022, September 15\textsuperscript{th}}

\begin{document}

\maketitle

\tableofcontents

\includepdf[pages={2}]{math6301a2-2022.pdf}

% Problem 1 ----------------------------------------------
\newpage
\section{}
\Problem
Given two metric spaces $(X,d)$ and $(Y,p)$, $a \in X$, and a function $f \st X \backslash \{a\} \to Y$.
We denote by $A_a(f)$ the set of all accumulated values of $f$ at $a$.
Show that $A_a(f)$ is a closed set.

\Preliminaries
\begin{definition}
    Let $(X,d)$ be a metric space.
    The set $A \subset X$ is called \emph{\underline{open}} if \[
        \forall_{a \in A} \exists_{\epsilon} \st \forall_{x \in X} d(a,x) < \epsilon \implies x \in A
    \]
\end{definition}
\begin{definition}
    A set is \emph{\underline{closed}} if it is not open.
\end{definition}
\begin{definition}
    Let $(X,d)$ and $(Y,p)$ be metric spaces and $a \in X$. 
    The function $f \st X \backslash \{a\} \to Y$ has an \underline{\emph{accumulation value (cluster value)}} $b \in Y$ if there exists a sequence $\{x_n\} \subset X \ \{a\}$ such that $\lim_{n \to \infty} x_n = a$ and $\lim_{n\to\infty} f(x_n) = b$.
\end{definition}

\Solution
\begin{proposition}
    The set of all accumulation points, $A_a(f)$, is closed.
\end{proposition}
\begin{proof}
    $A_a(f)$ is closed if it contains all of its limit points.

    If $A_a(f)$ is unbounded, we have $A_a(f) = Y$, which is closed.

    If $A_a(f)$ is empty, we have $A_a(f) = \emptyset$, which is also closed.

    If $A_a(f)$ is bounded and nonempty, then in order for it to be closed, it's upper and lower bound must be within the set.
    So we must show that $\sup A_a(f)$ and $\inf A_a(f)$ are included within the set.
    By definition, $\sup A_a(x) = \limsup_{x \to a} f(x)$ and $\inf A_a(x) = \liminf_{x \to a} f(x)$.
    Clearly $A_a(x)$ will contain both of these cases and as a result it is closed.
\end{proof}

% Problem 2 ----------------------------------------------
\newpage
\section{}
\Problem
Assume that $(X,d)$ is a metric space, $a \in X$ and $f \st X \backslash \{a\} \to \R$ is a function.
Put for $\delta > 0$ \[
    C_\delta(a) \coloneqq \{x \in X \st 0 < d(x,a) < \delta\}
\] Show that \[
    \limsup_{x \to a} f(x) = \inf_{\delta > 0} \sup_{x \in C_\delta(a)} f(x)
\]\[
    \liminf_{x \to a} f(x) = \sup_{\delta > 0} \inf_{x \in C_\delta(a)} f(x)
\]

\Solution

Recall that for a given sequence $\{y_n\} \subset \R$ we have \[
    \limsup_{n\to\infty} y_n = \inf_{n} \sup_{k \geq n} y_k = \lim_{n\to\infty} (\sup_{k\geq n} y_k)
\] and \[
    \liminf_{n\to\infty} y_n = \sup_{n} \inf_{k \geq n} y_k = \lim_{n\to\infty} (\inf_{k\geq n} y_k)
\]

We can then prove that the following propositions are true by showing that a sequence $y_k$ exists to satisfy these equalitites.

\begin{proposition}
    \[\limsup_{x \to a} f(x) = \inf_{\delta > 0} \sup_{x \in C_\delta(a)} f(x)\]
\end{proposition}
\begin{proof}
    Looking at the definition of $C_\delta(a)$, we see that we can construct every point $x_n,x_k \in C_\delta(a)$ so that $d(a, x_{n}) \leq d(a,x_{k})$ whenever $k \geq n$.

    Returning to the sequences that defined $A_a(f)$, we have that a cluster point $b$ exists iff the sequence $\{x_n\}$ exists with $b = \lim_{x \to a} f(x)$.
    We define a similar sequence $\{y_n\}$ where $y_n = f(x_n)$.
    In this sequence we now have that $\limsup_{n\to\infty} y_n = \lim_{n\to\infty} (\sup_{k\geq n} y_k)$.
    
    From $\{y_k\} \subset C_\delta(a)$ we have $\sup_{x \in C_\delta} f(x) = \sup_{k\geq n} y_k$

    Since $\limsup_{n\to\infty} y_n = \limsup_{x \to a} f(x)$ and $\{y_k\} \subset C_\delta(a)$, we can apply $\limsup_{n\to\infty} y_n = \inf_{n} \sup_{k \geq n} y_k = \lim_{n\to\infty} (\sup_{k\geq n} y_k)$ to show that \[\limsup_{x \to a} f(x) = \inf_{\delta > 0} \sup_{x \in C_\delta(a)} f(x)\]
\end{proof}

The dual of this problem is proved similarly.
\begin{proposition}
    \[\liminf_{x \to a} f(x) = \sup_{\delta > 0} \inf_{x \in C_\delta(a)} f(x)\]
\end{proposition}
\begin{proof}
    Looking at the definition of $C_\delta(a)$, we see that we can construct every point $x_n,x_k \in C_\delta(a)$ so that $d(a, x_{n}) \leq d(a,x_{k})$ whenever $k \geq n$.

    Take $\{y_n\}$ where $y_n = f(x_n)$ which implies $\liminf_{n\to\infty} y_n = \lim_{n\to\infty} (\inf_{k\geq n} y_k)$.
    
    From $\{y_k\} \subset C_\delta(a)$ we have $\inf_{x \in C_\delta} f(x) = \inf_{k\geq n} y_k$

    Since $\liminf_{n\to\infty} y_n = \liminf_{x \to a} f(x)$ and $\{y_k\} \subset C_\delta(a)$, we can apply $\liminf_{n\to\infty} y_n = \lim_{n\to\infty} (\inf_{k\geq n} y_k)$ to show that \[\liminf_{x \to a} f(x) = \sup_{\delta > 0} \inf_{x \in C_\delta(a)} f(x)\]
\end{proof}


% Problem 3 ----------------------------------------------
\newpage
\section{}
Given a metric space $(X,d)$, $a \in X$, and a function $f \st X \backslash \{a\} \to \R$ such that $A_a(f) \neq \emptyset$ is also bounded. 

\Preliminaries
For given functions $f \st X \to \R$, excluding the point $a \in X$ will results in being able to study accumulation points $A_a(f)$.

\subsection{}
\Problem
Show that $\limsup_{x \to a} f(x) < \alpha$ for some $\alpha \in \R$ iff \[
    \exists_{\delta>0} \exists_{\overline{\alpha}< \alpha} \forall_{x \in X} 0 < d(x,a) < \delta \implies f(x) \leq \overline{\alpha}
\]
\Solution
Essentially, looking at $A_a(f)$, we have that there exists an upper bound, $\sup A_a(f)$, which aligns with this definition.
Looking at $A_a(f)$ the $\neq \emptyset$ and bounded align with $\forall_{x \in X} 0 < d(x,a) < \delta$ and the $f(x) \leq \overline{\alpha}$ is the upper bound of $A_a(f)$.

\subsection{}
\Problem
Show that $\liminf_{x \to a} f(x) > \alpha$ for some $\alpha \in \R$ iff \[
    \exists_{\delta>0} \exists_{\overline{\alpha} > \alpha} \forall_{x \in X} 0 < d(x,a) < \delta \implies f(x) \geq \overline{\alpha}
\]
\Solution
Essentially, looking at $A_a(f)$, we have that there exists a lower bound, $\inf A_a(f)$, which aligns with this definition.
Looking at $A_a(f)$ the $\neq \emptyset$ and bounded align with $\forall_{x \in X} 0 < d(x,a) < \delta$ and the $f(x) \geq \overline{\alpha}$ is the lower bound of $A_a(f)$.

\subsection{}
\Problem
Show that $\limsup_{x \to a} f(x) \leq \alpha$ for some $\alpha \in \R$ iff \[
    \forall_{\alpha' > \alpha} \exists_{\delta>0} \forall_{x \in X} 0 < d(x,a) < \delta \implies f(x) \leq \alpha'
\]
\Solution
Essentially, looking at $A_a(f)$, we have that there exists an upper bound, $\sup A_a(f)$, which aligns with this definition.
Looking at $A_a(f)$ the $\neq \emptyset$ and bounded align with $\forall_{x \in X} 0 < d(x,a) < \delta$ and the $f(x) \leq \overline{\alpha}$ is the upper bound of $A_a(f)$.


\subsection{}
\Problem
Show that $\liminf_{x \to a} f(x) \leq \alpha$ for some $\alpha \in \R$ iff \[
    \forall_{\alpha' < \alpha} \exists_{\delta>0} \forall_{x \in X}  0 < d(x,a) < \delta \implies f(x) \geq \alpha'
\]
\Solution
Essentially, looking at $A_a(f)$, we have that there exists a lower bound, $\inf A_a(f)$, which aligns with this definition.
Looking at $A_a(f)$ the $\neq \emptyset$ and bounded align with $\forall_{x \in X} 0 < d(x,a) < \delta$ and the $f(x) \geq \overline{\alpha}$ is the lower bound of $A_a(f)$.

% Problem 4 ----------------------------------------------
\newpage
\section{}
\Problem
Given two metric spaces $(X,d)$ and $(Y,p$), $a \in X$ and a function $f \st X \to Y$.
Show that $f$ is continuous iff \[
    A_a(f) = \{f(a)\}
\]
\Solution
\begin{proposition}
    Given two metric spaces $(X,d)$ and $(Y,p$) and  $a \in X$ and a function $f \st X \to Y$.
    $f$ is continuous iff \[
        A_a(f) = \{f(a)\}
    \]
\end{proposition}
\begin{proof}
    First we look at implication, $\implies$.

    $f$ being continuous means that \[
        \forall_{\epsilon > 0} \exists_{\delta > 0} \forall_{x \in X} d(x,a) < \delta \implies p(f(x), f(a)) < \epsilon
    \] or equivalently,\[
        \forall_{\{x_n \in X\}} \st \lim_{n \to \infty} x_n = a \implies \lim_{n \to \infty} f(x_n) = f(a)
    \]

    Clearly, the limit definition of continuity lines directly with the definition of the accumulation points.
    $A_a(f)$ being every point that $\lim_{x \to a} f(x) = b$ means that if the $\lim_{n \to \infty} f(x_n) = f(a)$ then there is only one point $b$ satisfying the accumulation point definition; and therefore, $A_a(f) = \{f(a)\}$.

    Next we look at $\impliedby$.
    $A_a(f) = \{f(a)\}$ means that for every sequence $\{x_n\}$ with $\lim_{n \to \infty} x_n = a$ the only accumulation point is $f(a)$, $\lim_{n \to \infty} f(x_n) = b = f(a)$.

    This is directly aligned with the limit definition of continuity, so \[
        \forall_{\{x_n \in X\}} \st \lim_{n \to \infty} x_n = a \implies \lim_{n \to \infty} f(x_n) = b = f(a)
    \]
\end{proof}



% Problem 5 ----------------------------------------------
\newpage
\section{}
Denote by $\overline{\mathbb{R}}$ the ordered set of real numbers $\{-\infty\} \cup \R \cup \{\infty\}$ and define the function $\phi \st \overline{\mathbb{R}} \to \R$ by \[
    \phi(x) = \begin{cases}
        \text{arctan}(x) & x \in \R\\
        \pm \frac{\pi}{2} & x = \pm \infty
    \end{cases}
\] and the function $d \st \overline{\R} \cross \overline{\R} \to \R$ by \[
    d(x,y) \coloneqq \abs{\phi(x) - \phi(y)}
\]

\subsection{}
\Problem
Show that the function $d$ is a metric on $\overline{\R}$.
\Solution
\begin{enumerate}
    \item \textbf{Positivity:} $d(x,y) \geq 0$ and $d(x,y) = 0$.
    
        By definition of absolute value, $\abs{\phi(x) - \phi(y)} \geq 0$.
        Additionally, by definition, $\abs{\phi(x) - \phi(x)} = \abs{0} = 0$

    \item \textbf{Symmetry:}  $d(x,y) = d(y,x)$.

        By definition of absolute value, we have $d(x,y) = \abs{\phi(x)  - \phi(y)} = \abs{-(\phi(y) - \phi(x))} = \abs{\phi(y) - \phi(x)} = d(y,x)$.

    \item \textbf{Triangle Inequality:} $d(x,z) \leq d(x,y) + d(y,z)$
    
        We must prove that \[
            d(x,z) = \abs{\phi(x) - \phi(z)} \leq \abs{\phi(x) - \phi(y)} + \abs{\phi(y) - \phi(z)} = d(x,y) + d(y,z)
        \]
        Since $\phi(x)$ is an always increasing function, $x > y \implies \phi(x) > \phi(y)$, for all cases of $z < x < y$, $x < z < y$, and $ x < y < z$ we have the simple triangle inequality from absolute values of difference hold.
\end{enumerate}

\subsection{}
\Problem
Show that the topology $\mathcal{T}$ induced by the metric $d$ on $\overline{\R}$, restricted to $\R$, coincide with the usual topology on $\R$.
\Solution

The metric topology induced for both cases is just the union of the open balls on $\R$.

Since the open balls for both metrics, defined by $B_a(r) = \{x \in \R \st d(a,x) < r\}$, have essentially the same form, it is clear that they are equivalent topologies.
(i.e) \[
    B_a(r) = \{x \in \R \st \abs{a - x} < r\} \approx \{x \in \R \st \abs{\arctan(a) - \arctan(x)} < r\}
\]

\end{document}
